\section{Result 3: Individual Parameter Sensitivity near the Global Optimum}\label{sec:GA:IndividualSensA}


\subsection{Individual Parameter Perturbation Analysis}

To further understand how useful each cost function was in
constraining parameters, a sensitivity analysis on each individual
parameter is is crucial to understand the behaviour of individual
parameters close to the global optimum.  The sensitivity analysis of
the cost function is defined as calculating the learning gradients of
each parameter on either side of the target value. Parameter values
were stepped up and down independently, with the steps determined from
the gene resolution of the parameter in
Table~\ref{tab:GA:Genome}. Gradients were calculated using a
least-squares linear regression in MATLAB/GNU Octave and two-sided
t-tests were performed to determine whether each gradient was
significantly different from zero.  This was done for the identical
and the different ANF inputs, robustness was evaluated by comparing
the ratio of V-shaped to non-V-shaped cost function gradients for
different inputs.

\medskip{}

Representative examples are given in Figure \ref{Fig:R4}, which show
the dependence of the cost function on perturbation size when a parameter was
perturbed from its target value (and all other parameters had their target
value). A range of different behaviors is evident depending on the particular
combination of parameter and cost function. The ideal behaviour is shown in
Figure \ref{Fig:R4}A, which shows the target at a well defined local minimum in
the cost function, with significantly non-zero gradients bilaterally. Figure
\ref{Fig:R4}B is a sub-ideal case, with a significantly non-zero gradient
appearing only unilaterally, a zero gradient on the opposing side.  The
behaviour shown in Figure \ref{Fig:R4}C, in which the cost function is locally
flat, implies that the cost function is insensitive to this parameter in the
vicinity of the target and represents non-ideal behaviour. Finally, Figure
\ref{Fig:R4}D gives an example of a problematic cost function behaviour, in
which the minimum occurs at a value other than the target.\todo[inline]{Hamish: I am still not
happy we understand this well: Pruning of candidates with reduced spikes in
low rate regions of DS units}. These cases were classified variously as
bilaterally sensitive (Figure \ref{Fig:R4}A), unilaterally sensitive (Figure
\ref{Fig:R4}B), insensitive (Figure \ref{Fig:R4}C) or irregular(Figure
\ref{Fig:R4}D), respectively.

\todo[inline]{ Correlation between relative param error and gradient in Fig 5 to be discussed }

% The gradients of the cost function above and below the target value are plotted
% in Figure ? for each individual parameter and for the three different cost
% functions. {\bf order and comment on similarities. Also comment on correlation
% between parameter sensitivity and parameter error.}
% 
% A summary of these data are given in Table ?, which compares the cost
% function on basis of how many parameters showed sensitivity that was
% bilateral, unilateral or absent, or contained opposing gradients.

% I would be better to present these results in table comparing
% For different ANF
% inputs (Figure~\ref{fig:13}B), 11 parameters were bilaterally
% sensitive and 8 were unilaterally sensitive, while the ST cost
% function was completely insensitive to 6 parameters. Three parameters
% controlling the excitatory synaptic input to DS cells, 4, 5 and 6
% ($w_{{\rm ANF}\to {\rm DS}} $, $n_{{\rm HSR}\to {\rm DS}} $, $n_{{\rm
% LSR}\to {\rm DS}} $) had significant opposing gradients below the
% target, and the inhibitory input parameter 25 ($n_{{\rm TV}\to {\rm
% DS}} $) above the target suggesting a shifted optimum value.



 \begin{figure}[thp!]
   \centering
   \includegraphics[width=\textwidth]{./gfx/Example_SensAnalysis.eps}  
  \caption{Examples of ST cost function sensitivity analysis
    performed on individual parameters, with 10 unit step increments
    around the parameter's target value and all other target
    parameters retained. Multiple samples were taken at each point
    when different inputs were used. The linear regression line
    (solid) and bootstrapped 95\% confidence interval (dotted line)
    are shown. The slope was tested for significant difference to a
    zero gradient either side of the target value. (A) Parameter 3,
    $n_{{\rm HSR}\to {\rm TS}} $, was V-shaped for identical and
    different inputs. (B) Parameter 4, $w_{{\rm ANF}\to {\rm DS}} $,
    was V-shaped for identical inputs, but for different inputs the
    gradient below the target was significantly opposed to the
    correct direction. (C) Sensitivity around parameter 23, $s_{{\rm
        DS}\to {\rm TV}} $, was V-shaped for identical inputs but
    only one gradient was significant for different inputs. (D) The
    sensitivity of the ST cost function around parameter 29,
    $n_{{\rm GLG}\to {\rm DS}} $, produced the largest V-shaped
    gradients for identical and different inputs.}
  \label{fig:12}
\end{figure}



\begin{figure}[htb]
  \centering
  \caption{Parameter sensitivity gradient plots for the ST cost
    function with ideal input (A) and with different ANF input
    (B). Parameter gradients that are significantly different from
    zero (Student's t-test p $<$ 0.05) are shown with asterisk
    ($\ast$) and error bars that are the standard error of the
    slope. Gradients that are opposite to expected are shown in
    solid bars, with significant difference shown with a diamond
    ($\diamond$).}
  \label{fig:13}
\end{figure}



\begin{figure}[htb]
  \centering
  \caption{Parameter sensitivity gradient plots for the IFR cost
    function with the format similar to Figure~\ref{fig:13}. (A) The
    IFR-25 cost function with identical input. (B) The IFR-25 cost
    function with different ANF inputs. (C) The IFR-100 cost
    function with different ANF inputs.}
  \label{fig:14}
\end{figure}



\begin{figure}[htb]
  \centering
  \caption{Parameter sensitivity gradient plots for the AIV cost
    function with the format similar to Figure~\ref{fig:13}. A The
    AIV-25 cost function with identical input. (B) The AIV-25 cost
    function with different ANF inputs. (C) The AIV-100 cost
    function with different ANF inputs.}
  \label{fig:15}
\end{figure}




\subsubsection{Spike Timing}

\subsubsection{Instantaneous Firing Rate}

\subsubsection{Average Intracellular Voltage}


\subsubsection{Best Genome Match to Individual Parameter Sensitivity}

% 
% When the ANF input was slightly different from the ideal input, the optimum
% increases and the gradient of parameter 1, $w_{{\rm ANF}\to {\rm TS}}
% $, diminishes (Figure~\ref{fig:12}C). The spread of connections from
% DS cells to TV cells is wide (target value=8 channels), covering one
% third of the network (30 channels) so the non-linear jumps could be
% due to random selection of pre-synaptic cells or confounding effects
% of TV cells on TS and DS cells. With different inputs, parameter 23
% ($s_{{\rm DS}\to {\rm TV}} $) sensitivity of the ST cost function was
% robust below the target but is negative above the target although not
% significantly. Gradients that oppose the direction toward the target
% would reduce the effectiveness of optimization, especially
% gradient-decent methods.  For parameters with a target value close to
% the minimum range (parameters 17, 20, 26, and 27), the gradient below
% the target were not considered in the sensitivity analysis.
% 
% DS cells have very precise onset spikes and few spikes in the remainder
% of the stimulus.  If the weight and number of excitatory inputs were
% reduced, the spike timing difference would not be influence by (or
% inhibitory input increased), the onset spikes would still occur but
% the larger difference in the random positions the number of spikes in
% the sustained period of the stimulus would be reduced and the there
% would be some benefit to this change in the training data.
% 
% The sensitivity to inhibitory parameters of the
% IFR-25 cost function was not robust to changes in the ANF input
% (Figure~\ref{fig:14}B) since most gradients were flattened (not
% significant from zero gradient) or were unilaterally sensitive.  Seven
% inhibitory parameters had opposing gradients below the target, but
% only one was significant, 14 ($s_{{\rm DS}\to {\rm TS}} $). Parameter
% 10 had a significant reduction from identical inputs to different
% inputs, where it became completely insensitive.





%%% Local Variables: 
%%% mode: latex
%%% TeX-master: "GAChapter"
%%% TeX-PDF-mode: nil
%%% End: 
