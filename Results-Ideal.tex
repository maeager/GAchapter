\section{Optimisation of BNNs using Ideal inputs}

To understand the of optimising BNNs it may appear that to use ideal
inputs is not intuitive; however, the methods and techniques of GA
optimisation in this chapter were initially refined using an ideal
environment.

\subsection{Genetic Algorithm Performance}


 \todo[inline]{Summary of Ideal Input GA performance}
% 
% Faster evolution?? Does not look like it to me.
% 
% Cost function scores for the best genomes emerging from the GAs in
% the absence of noise are given in row 2 of Table ? for all three
% cost functions. For ease of comparison the equivalent scores in the
% case with noisy inputs are repeated in row 1.  In general, across
% cost functions, use of ideal input led a lower score for the best
% genome than was the case when noise was present. On the other hand,
% no best genome came close to obtaining an error-free score of zero.
% 
% The parameter sensitivity analyses provide insight into this
% result. Results from the 1 unit step and 5 unit step simultaneous
% parameter perturbation analysis are given in Figure ? for the
% scenario of ideal inputs. In general, they show that while the
% target had the expected error-free score of zero, 1 unit step and 5
% unit step perturbations both lead to scores that were considerably
% above zero.  This suggest that even the smallest perturbation leads
% to a discontinuous jump in the cost function. In general, it can
% also be seen that score obtained by best genome corresponds
% approximately to the mode of the 5 unit step distribution of scores
% and approaches the range of scores obtained from 1 unit step
% perturbations. This suggests that the GA was able to perform
% reasonably well up to the point at which the cost function became
% discontinuous (i.e. at the target).
% 
% This conclusion is supported by the individual parameter sensitivity
% analysis (Figure ?) which shows that some parameters gave rise to
% large jump discontinuities in the cost function at the target
% value. These parameters were typically the number of synaptic
% connections from one neural type to another. As such there were
% discrete and {\bf need some help here about what actually happened}.
% 
% Table ? provides a statistical summary of the individual parameter
% sensitivity analysis, with rows 1 and 2 comparing the analysis for the
% noisy and ideal input scenarios. For ideal inputs, the vast majority of
% parameters showed significant bilateral sensitivity, regardless of the
% cost function, whereas in the noisy case only 50\% or less did.
% 
% 
% 
% 
% {\it Comment: Need to say something about the match to target parameters.}





\subsection{Parameter space sensitivity of cost functions}


\subsection{Performance of best genomes and comparison of cost functions }
