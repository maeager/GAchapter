\section{Optimisation of BNNs using Ideal inputs \textbf{OR} Expanded results}

To understand the of optimising BNNs it may appear that to use ideal
inputs is not intuitive; however, the methods and techniques of GA
optimisation in this chapter were initially refined using an ideal
environment.

\todo[inline]{This is an attempt to include excess material into the
  thesis that was chucked out after the failure of the JNeuroPhysiol
  submissions. Anything with 250 repetitions has been removed}


\subsection{Genetic Algorithm Performance}


\todo[inline]{Summary of Ideal Input GA performance}

The performance of the GA optimization is illustrated by the evolution
of the population of genome scores (Figures~\ref{fig:GA:5}
to~\ref{fig:GA:7}) and by the best score in each generation. The
evolutions of the population scores are represented in
Figure~\ref{fig:GA:5} by the 25--75 percentile range of scores in each
generation (shaded area). The best genome score in each generation
(solid line) shows the different learning phases of the GA, from large
steps initially to more incremental improvements as the GA tends
towards an asymptote. The parameter error between the best genome's
parameters and the target parameters are shown in Figure~\ref{fig:GA:8},
a combined parameter error is calculated by normalizing each parameter
by its range and finding the mean absolute error.

\smallskip{}

\begin{figure}[htb]
\figfont{A}\hspace{2.2in}\figfont{B} \hfill \\
\resizebox{5in}{!}{\includegraphics{STDYN25NormGAPerf.eps}\hspace{1cm}\includegraphics{STDYN25DiffANGAPerf.eps}}\hfill\\
 \caption{ST cost function GA performance for ideal (A) and different (B) inputs.}\label{fig:GA:5}
\end{figure}

\begin{figure}[ht!]
\figfont{A}\hspace{2.2in}\figfont{B} \hfill \\
\resizebox{5in}{!}{\includegraphics{IFRGA25NormGAPerf.eps}\hspace{1cm}%
\includegraphics{IFRGA25DiffANGAPerf.eps}}\hfill\\
 \caption{}\label{fig:GA:6}
\end{figure}

\begin{figure}[ht!]
\figfont{A}\hspace{2.2in}\figfont{B} \hfill \\
\resizebox{5in}{!}{\includegraphics{IVGA25NormGAPerf.eps}\hspace{1cm}%
\includegraphics{IVGA25DiffANGAPerf.eps}}\hfill\\
 \caption{}\label{fig:GA:7}
\end{figure}



 \begin{figure}[thb!]
%  \psfrag{0030}[br][br][1][0]{${s}_{GLG\rightarrow{DS}}$}
%  \psfrag{0029}[br][br][1][0]{${n}_{GLG\rightarrow{DS}}$}
%  \psfrag{0028}[br][br][1][0]{${w}_{GLG\rightarrow{DS}}$}
%  \psfrag{0027}[br][br][1][0]{${o}_{DS\rightarrow{TV}}$}
%  \psfrag{0026}[br][br][1][0]{${s}_{TV\rightarrow{DS}}$}
%  \psfrag{0025}[br][br][1][0]{${n}_{TV\rightarrow{DS}}$}
%  \psfrag{0024}[br][br][1][0]{${w}_{TV\rightarrow{DS}}$}
%  \psfrag{0023}[br][br][1][0]{${s}_{DS\rightarrow{TV}}$}
%  \psfrag{0022}[br][br][1][0]{${n}_{DS\rightarrow{TV}}$}
%  \psfrag{0021}[br][br][1][0]{${w}_{DS\rightarrow{TV}}$}
%  \psfrag{0020}[br][br][1][0]{${s}_{GLG\rightarrow{TS}}$}
%  \psfrag{0019}[br][br][1][0]{${n}_{GLG \rightarrow{TS}}$}
%  \psfrag{0018}[br][br][1][0]{${w}_{GLG\rightarrow{TS}}$}
% \psfrag{0017}[br][br][1][0]{${s}_{TV\rightarrow{TS}}$}
% \psfrag{0016}[br][br][1][0]{${n}_{TV\rightarrow{TS}}$}
% \psfrag{0015}[br][br][1][0]{${w}_{TV\rightarrow{TS}}$}
% \psfrag{0014}[br][br][1][0]{${s}_{DS\rightarrow{TS}}$}
% \psfrag{0013}[br][br][1][0]{${n}_{DS\rightarrow{TS}}$}
% \psfrag{0012}[br][br][1][0]{${w}_{DS\rightarrow{TS}}$}
% \psfrag{0011}[br][br][1][0]{${n}_{LSR\rightarrow{GLG}}$}
% \psfrag{0010}[br][br][1][0]{${w}_{LSR\rightarrow{GLG}}$}
% \psfrag{0009}[br][br][1][0]{${n}_{HSR\rightarrow{TV}}$}
% \psfrag{0008}[br][br][1][0]{${n}_{LSR\rightarrow{TV}}$}
% \psfrag{0007}[br][br][1][0]{${w}_{ANF\rightarrow{TV}}$}
% \psfrag{0006}[br][br][1][0]{${n}_{HSR\rightarrow{DS}}$}
% \psfrag{0005}[br][br][1][0]{${n}_{LSR\rightarrow{DS}}$}
% \psfrag{0004}[br][br][1][0]{${w}_{ANF\rightarrow{DS}}$}
%  \psfrag{0003}[br][br][1][0]{${n}_{HSR\rightarrow{TS}}$}
%  \psfrag{0002}[br][br][1][0]{${n}_{LSR\rightarrow{TS}}$}
% \psfrag{0001}[br][br][1][0]{${w}_{ANF\rightarrow{TS}}$}
% \psfrag{H}[br][br][1][0]{\figfont{\Large{H}}}
% \psfrag{G}[br][br][1][0]{\figfont{\Large{G}}}
% \psfrag{F}[br][br][1][0]{\figfont{\Large{F}}}
% \psfrag{E}[br][br][1][0]{\figfont{\Large{E}}}
% \psfrag{D}[br][br][1][0]{\figfont{\Large{D}}}
% \psfrag{C}[br][br][1][0]{\figfont{\Large{C}}}
% \psfrag{B}[br][br][1][0]{\figfont{\Large{B}}}
% \psfrag{A}[br][br][1][0]{\figfont{\Large{A}}}
 \resizebox{5in}{!}{\includegraphics{BestGenomes-4.0.eps}}
 \caption{}
     \label{fig:GA:8}
 \end{figure}


For the ST cost function with identical ANF inputs
(Figure~\ref{fig:GA:5}A) the population scores were initially spread over
a wide range of values. As the GA progressed there was rapid
improvement in the first 50 generations. The results then asymptote to
a mean score around 30 ms per spike train, although there was
fluctuation throughout the remaining generations.  The best score
after 200 generations was 8.45 ms with the best genome steadily
improving until the final generation.  The GA run using the ST cost
function and different ANF inputs (Figure~\ref{fig:GA:5}B) had a similar
learning profile, but there was less variability in the 25--75
percentile range in the later generations and the best genome score
was 9.72~ms (Figure~\ref{fig:GA:5}B).  The best genome for the identical
inputs was also closer to the target parameter values shown in
Figure~\ref{fig:GA:8}A, with a normalized mean parameter error of 0.221,
while the different inputs GA's best genome was 0.252 (Figure
8B). Some parameters were well constrained by the GA and were robust
to changes in the input, such as the excitatory input corresponding to
the ANF input to the CN cells (parameters 1 to 11 or \wANFTS to
\wLSRGLG) and some inhibitory parameters (12, 18, and 20 corresponding
to \wDSTS, \wGLGTS, and \sGLGTS respectively).

\smallskip{} 

The GA was run with different combinations of the IFR cost function,
first using 25 repetitions in the training data (IFR-25) with
identical ANF inputs in the GA evaluation, secondly using IFR-25 with
different ANF inputs, and lastly using 100 repetitions in the training
data (IFR-100) with different ANF inputs. Figure~\ref{fig:GA:6}A, shows
the GA performance of the IFR-25 cost function with identical
inputs. The range of the 25--75 percentile population evolved quite
rapidly before settling between 0.3~and 0.25~sp/ms.  The histogram of
evaluated scores peaks around 0.25~sp/ms with a tail toward 0.2 sp/ms.
The best genome's score of 0.195~sp/ms is equivalent to an average
PSTH error of 11.8~spikes per cell. In terms of the parameter error
from the target, the IFR-25 cost function with identical inputs
returned the closest genome to the target of 0.201
(Figure~\ref{fig:GA:8}C) for all GA simulations. When the inputs were
randomized and the training data (25~reps) remained the same, the GA
populations' learning was considerably slower and the search space was
more compact, Figure 6B. \todo[inline]{linkback to previous
  section}. This meant that there was less difference between a good
genome and a bad genome.  The best genome obtained by the IFR-25 cost
function with different inputs had a score of 0.263~sp/ms and a mean
parameter error of 0.273 (Figure~\ref{fig:GA:8}D). The GA run using the
IFR-100 cost function with different inputs shifted the general
population of GA scores lower than the IFR-25 cost function, with
population scores between 0.25~and 0.15~sp/ms. The learning was rapid
in the first 50 generations but reached a steady state and the best
genome score was not improved beyond the 150th generation
(Figure~\ref{fig:GA:6}C).  %The best genome's obtained the worst mean
%parameter error of 0.297~for all GA simulations (Figure~\ref{fig:GA:8}E).

The GA performance was similar for each of the AIV cost functions
conditions in Figure~\ref{fig:GA:7}. The initial population of each AIV
cost function method ranged from 0.5~to 0.4~in the 25-75\% population
score, with a rapid learning phase in the first 50 generations and a
gradual learning phase and a smooth distribution of scores.  The
AIV-25 cost function with identical ANF inputs produced the lowest AIV
cost function score, 0.151~mV (Figure~\ref{fig:GA:7}A).  The AIV-25 and
AIV-250 cost functions with different inputs scored, 0.208~and 0.188
mV, respectively.  The mean parameter errors of the best genome for
the AIV-25 cost function with identical inputs, the AIV-25 cost
function with different inputs and the AIV-250 cost function with
different inputs were, 0.258, 0.207 and 0.275, respectively (Figure
8F-H).  The performance of the best genome generated by the AIV-25
cost function with different inputs was very accurate for inhibitory
parameters (Figure~\ref{fig:GA:8}G) presumably due to subthreshold
information within the intracellular voltages. \todo[inline]{remove or replace 250 with 100}


% 
% Faster evolution?? Does not look like it to me.
% 
% Cost function scores for the best genomes emerging from the GAs in
% the absence of noise are given in row 2 of Table ? for all three
% cost functions. For ease of comparison the equivalent scores in the
% case with noisy inputs are repeated in row 1.  In general, across
% cost functions, use of ideal input led a lower score for the best
% genome than was the case when noise was present. On the other hand,
% no best genome came close to obtaining an error-free score of zero.
% 
% The parameter sensitivity analyses provide insight into this
% result. Results from the 1 unit step and 5 unit step simultaneous
% parameter perturbation analysis are given in Figure ? for the
% scenario of ideal inputs. In general, they show that while the
% target had the expected error-free score of zero, 1 unit step and 5
% unit step perturbations both lead to scores that were considerably
% above zero.  This suggest that even the smallest perturbation leads
% to a discontinuous jump in the cost function. In general, it can
% also be seen that score obtained by best genome corresponds
% approximately to the mode of the 5 unit step distribution of scores
% and approaches the range of scores obtained from 1 unit step
% perturbations. This suggests that the GA was able to perform
% reasonably well up to the point at which the cost function became
% discontinuous (i.e. at the target).
% 
% This conclusion is supported by the individual parameter sensitivity
% analysis (Figure ?) which shows that some parameters gave rise to
% large jump discontinuities in the cost function at the target
% value. These parameters were typically the number of synaptic
% connections from one neural type to another. As such there were
% discrete and {\bf need some help here about what actually happened}.
% 
% Table ? provides a statistical summary of the individual parameter
% sensitivity analysis, with rows 1 and 2 comparing the analysis for the
% noisy and ideal input scenarios. For ideal inputs, the vast majority of
% parameters showed significant bilateral sensitivity, regardless of the
% cost function, whereas in the noisy case only 50\% or less did.
% 
% 
% 
% 
% {\it Comment: Need to say something about the match to target parameters.}


\subsection{Parameter space sensitivity of cost functions}


\subsubsection{Spike Timing}
\subsubsection{Instantaneous Firing Rate}
\subsubsection{Average Intracellular Voltage}

% The distribution of 1 step and 5 step parameter variations was
% separated with identical inputs but was still significantly different
% for simulations with different inputs.


\subsection{Performance of best genomes and comparison of cost functions }

\todo[inline]{refine J Neurophysiol section to go here}



%%% Local Variables: 
%%% mode: latex
%%% TeX-master: "GAChapter"
%%% TeX-PDF-mode: nil
%%% End: 
