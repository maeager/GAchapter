



%% * Discussion
%% ** General performance of GAs for optimising network parameters
%% ** Summary of cost functions
%% *** Spike Timing
%% *** Instantaneous Firing Rate
%% *** Average Intracellular Voltage
%% ** Effects of noise in BNN optimisation
%% *** Ideal or realistic neural input
%% *** Benefits of reducing noise by increased repetition
%% ** Sensitivity of parameters
%% ** Comparison with other studies
%% ** Other considerations for constraining BNNs
%% * Conclusion



\section{Discussion}\label{sec:GA:discussion}

\subsection{General performance of GAs for optimising network parameters}\label{sec:GA:general-perf}

We tested the ability of the GA to learn the network parameters of a
biophysically-based neural network (BNN) using three cost
functions. Figure~\ref{fig:R1} showed the typical behaviour of GA populations,
with a convergence of the best genome toward the minimum value and the noisy
variation in the population of genome scores. GA's run with the ST and IFR cost
functions were able to consistently achieve convergence of the best genome to
very near the target score; the deviation was less than that due to a 1 unit
step perturbation of the target genome. For GA's run with the AIV cost function,
however, convergence of best genomes was inconsistent and did not reach the
target. Instead the AIV scores were limited to the upper tail of the
distribution obtained with 5 unit step parameter perturbations of the target
genome.

\medskip{}

By increasing the number of repetitions in the GA evaluation, in
Figure~\ref{fig:R5}, the cost function scores were reduced by the reduction in
noise. The GA performance did not lead to significantly improved match to the
target genome, with the exception of the AIV cost function.  The cost function
scores were significantly reduced for IFR and AIV cost functions but GA
performance did not lead to significantly improved match to the target genome.


\medskip{}

keypoints:
* no basis for one CF better than another , 
* All performed similarly in most measures of comparison, X-comp and PE analysis
* X-comp: all best genomes performed similarly in IFR and AIV, but ST was better in it's won CF
* PE: similar trend/pattern similar PEvP in each given parameter (note significance of different parameter types - orders of 2 or 3 between spread/numberand weight parameters)

* noteworthy: ST Xcomp: expected result AIV Xcmomp: ST does not
perform poorly on AIV given different characteristics of CF,
i.e. spike timing does not provide information about
sub-threshold membrane potential, a relevant factor in the AIV
CF.  But this is good because, ST can be obtained in extra
cellular recordings more easily in vivo and can be obtained on a larger
number of simultaneous units than intracellular recordings.


*       IFR consistently off in ST and AIV
AIV CF also limited  ST and IFR best genomes to same limit as AV best genome 



\subsection{Summary of cost functions}\label{sec:GA:summ-cost-funct}

This section gives an overview of the cost functions' advantages, disadvantages,
and relevance to optimizing BNNs. The following summary of the cost functions
will highlight and compare the results by focusing on three main performance
measures.  Measure 1 indicates whether the best genome obtained a \textit{poor},
\textit{good}, or \textit{very-good} score. Very good scores are below the mean
of the 1-step deviation population, poor scores are above the mean of the 5-step
deviation population, and good scores are in between.  %Measure
% 2 indicates the robust sensitivity of the cost function and is the
% ratio of significant V-shaped sensitivity gradients with different AN
% inputs against other less sensitive parameters (i.e any shape that is
% significantly non-V-shaped or other V-shapes that are not
% significant). The gradients below the target value of parameters 17,
% 20, 26, and 27 were ignored, as explained in section
% 3.3~\textit{Individual Parameter Sensitivity near the Global Optimum}.
Measure 2 gives the relative geometric distance between the target
parameter set and the target genome's parameters.

\subsubsection{Spike timing}\label{sec:GA:spike-timing-summ}

The results of the ST cost function show that it can be successfully used for
optimising BNNs.  For Measure 1, the results were very-good for the best genome
with different inputs as it scored below the mean of target genome scores. It
also achieved a good score when evaluated with the other cost functions.  For
Measure 2, the mean geometric distance between the ST best genome and the target
was 0.252, the second best next to the AIV-25 cost function for GA simulations
run with different AN inputs.

\medskip{}

A major benefit in using spike times in optimisation of real networks is their
ease of collection \textit{in vivo}. Data from populations of neurons can be
obtained by extracellular single- or multi-unit recordings.  By sampling over
all cells multiple times, this method provides a good estimate of the temporal
information contained in the neural responses, enabling reasonable parameter
optimisation and good robustness to noise.  The key disadvantage associated with
spike train comparisons is the increased computational time associated with the
evaluation of the cost function score.  For $R=25$ repetitions, the ST cost
function took just as long to run as the CN network simulation time
(approximately 90 seconds in a single 2GHz CPU) depending on the level of
activity. Increasing the number of repetitions scaled the computation time by
$R^2$ due to the cross comparison of available spike trains in the training data
to find the minimum error.

\medskip{}

Noise was minimised by the comparison procedure that found the minimum score
among 25 spike trains in the training data. Any additional data from more
repetitions or more neurons may be beneficial for the robustness to noise, but a
combination with another source of data, for example AIV waveforms, would be
more suitable. The dynamic programming algorithm in the ST cost function is
similar to the temporal difference method of \citet{VictorGoldbergEtAl:2007},
except that specific penalties were not applied to insertions and deletions of
spikes.

% The synaptic parameters have a strong influence on the timing of spikes in
% post-synaptic neurons, contributing to changes in the cost function score when
% the parameters are moved further away from the target, and provide a
% well-defined global optimum. Even though the number of individual parameters
% with significant learning gradients was reduced for the ST cost function with
% different inputs and there were four parameters with significant opposing
% gradients (Figure 13B), the cost function still produced a distinctive optimum
% (Figure 9B) and tqhe GA was able to find genomes close to the global optimum.

\subsubsection{Instantaneous Firing Rate }\label{sec:GA:inst-firing-rate-summ}

\todo[inline]{Hamish noted that this para was not applicable} When considering the 25
repetition IFR cost function's performance, Measure 1 for the best genome
obtained with different inputs was poor for all cost functions. When the number
of repetitions in the training data was increased by a factor of 4 (the IFR-100
cost function), there was a reduction in the value of all cost function scores
(Figure~\ref{fig:R6}B) and the performance of the best genomes was good (Figure
\ref{fig:R7}). However, the performance of both the IFR-25 and IFR-100 best
genomes when measured using the other cost functions was poor, this suggests
that the networks constrained by the IFR cost functions were unable to
accurately reproduce the behaviour of networks in terms of spike-times or
intracellular voltage. %The individual parameter sensitivity measure,
% Measure 2, gave a ratio of 7:23~and 9:21~for the IFR-25 and IFR-250
% cost functions, respectively (Figure~\ref{fig:14}B,C), demonstrating a
% high susceptibility to input noise. Constraint of inhibitory
% connections (parameters 12-30) in the IFR best genomes
% (Figure~\ref{fig:8}D,E) was very poor, resulting from the flat and
% insignificant cost function gradients near the global optimum (Figure
% 14B,C).
For Measure 2, the results of the IFR-25 and IFR-100 best genomes found using
different inputs show large mean parameter errors of 0.273~and 0.297,
respectively (Figure~\ref{fig:8}D-E).

\medskip{}

Grouping spike trains into time bins is a very fast procedure aimed at reducing
the trial-to-trial variability in single spike trains by generating an estimate
of the average instantaneous firing rate of neurons. The temporal resolution of
the IFR cost function is dependent on the width of the PSTH bins but it looses
information about the timing between spikes.  The representation of precise
onset spikes in DS and TS cells would benefit from a narrow bin width, but for
the majority of spikes in the network, the fine timing is not as important
during a noisy stimulus.  For the frozen notch noise, spatio-temporal peaks in
neural activity occur across the network and require enhanced temporal precision
in the IFR cost function, as shown in Figure~\ref{}.

\medskip{}

To improve the representation of firing-rate information, we must take into
consideration the width of the bins in a PSTH and their relationship to the
stochastic output of neurons.  It is desirable to have fine temporal resolution,
but the results of the IFR cost function show that the small bins are dominated
by noise, especially in low-firing units and in onset units apart from the first
spikes. A solution to this problem in future experiments would be to use
equi-probable bins in linear or log form \cite{BhumbraInyushkinEtAl:2004}.  This would improve the
performance of the IFR cost function by improving the sensitivity to changes in
parameters of the network.

\subsubsection{Average Intracellular Voltage}\label{sec:GA:aver-intr-volt-summ}

For Measure 1, the best genome constrained by the AIV-25 cost function was very
good when evaluated using all cost functions (Table~\ref{tab:5}) except for the
ST cost function for which it was good. The best genome trained using the
AIV-100 cost function was also very good for the ST, IFR-100, and AIV-25 cost
functions, and good for the IFR-25 and AIV-100 cost
functions.  % For Measure 2, the sensitivity
% ratios of individual parameters using different inputs were 13:17 and
% 9:21 for the AIV-25 and AIV-250 cost functions, respectively
% (Figure~\ref{fig:15}B,C).  This demonstrates that the AIV-25 cost
% function has greater robustness to noise than the AIV-250 and the IFR
% cost functions, and similar performance to the ST cost function.
For Measure 2, the parameter error of 0.207~for the AIV-25 best genome was the
best for all GA simulations that were run with different inputs in this study,
while the AIV-100 best genome was further from the target genome with an error
of 0.275 (Figure~\ref{fig:8}G,H).

\medskip{}
The average IV waveform over several repetitions aimed to reduce the effect of
trial-to-trial error and filter out APs.  Similar to the point-to-point method
comparison in the IFR cost function, increasing the number of repetitions
smoothed out the training data in the AIV-100 cost function scores and reduced
the scores for parameters close to the target (1-step and 5-step) reduced to the
level of the ideal scores (Figure~\ref{fig:11}A,C).

\medskip{}

It was thought that for a BNN model the average IV waveform will provide
additional information about the sub-threshold behavior of neurons in the
network, which is not available in the ST and IFR cost functions.  Intracellular
voltage data has been used in constraining the membrane conductances of
multi-compartmental single-neuron models \cite{Le_Masson:2000,KerenPeledEtAl:2005}.
These methods are not always effective in single simulations unless combined
with other cost functions, such as inter-spike intervals
\citep{KerenPeledEtAl:2005}. Phase-plane analysis of IV data was very effective
in optimizing membrane parameters \citep{VanDeEtAl:2008,KerenPeledEtAl:2005} but would not
be suitable to a BNN due to variation in the synaptic input and the loss of
temporal information.  It is not currently possible to obtain simultaneous IV
recordings from more than two neurons let alone a whole nucleus, but limited IV
data could be used in conjunction with other cost functions to constrain BNN
models.

\subsection{Benefits of reducing noise}\label{sec:GA:benef-reduc-noise}

\medskip{} 

\todo[inline]{ Must fill this out before sending back to hamish}
*No real differences in eventual performance
  despite reduction in score

** Lower cost function scores for all CFs

** Fig R7 Xcomp shows GA best genomes run with 25
    performed approx the same as GA's run with 100
** only clear exception being AIV 100 significantly better in ST CF

** noteworthy AIV limitations from 25 reps were removed
  in 100 reps, with the best genome's score were closer to the target
  (less than mean of 5 unit step perturbation).

\medskip{}

One of the big problems in optimizing BNNs is noise.  The various sources of
noise arise in the stochastic nature in neural transmission and connectivity
and in the algorithm chosen by the cost functions. Afferent input connections
and intrinsic connections within the microcircuit are defined by organised but
random connectivity.  Small perturbations in the parameters controlling the
number of inputs will change the selection of pre- and post-synaptic cells in
the construction of the network.  The smoothing of PSTH and IV also produces
inherent errors in the training data for parameters near the target
parameters, some of which perform better.  The main effect of noise in
optimization is over-fitting to the noise, resulting in a best genome scores
that are better than the target genome's distribution scores.  The GA run with
ST cost function and different inputs produced a score better than the mean
target with only one sample, when sampled multiple times the mean score was
also below the mean target scores but not statistically significantly.  In all
cost functions, the flattening of the cost function search spaces around the
target parameters contributed to an overlap between the 1-step population and
the distribution of the target genome.

\subsection{Comparison with other studies}\label{sec:GA:comp-with-other}


\todo[inline]{More work to be completed here}

GA could not have done an better than any other optimisation techniques.
Evolutionary vs Grad decent studies?  Consistency, efficiency for BNNs.

\medskip{}

*Studies with  other cost functions - do they get close to target? ISI CF studies?
Are there any studies showing ST and IFR/AIV? any comparisons?

\subsection{Other considerations for constraining BNNs}\label{sec:GA:other-considerations}


In this chapter limiting the number of parameters used to define the connectivity
of a BNN was critical for a practical method of optimization. Simplifying the
synaptic strength between two cell types to uniform weight and number
significantly reduced the number of parameters required for optimization, but
uniformity is unlikely for the real network weights.  A Gaussian weight
distribution is common among network models and would only add one parameter per
connection (i.e.\ standard deviation with the existing uniform mean parameter).
Optimizing conduction- and synaptic-delay is not covered in this paper, but
could add to further realism in BNN optimization.

\medskip{}

A final issue that should be considered for modelling and optimizing BNN models
is computational efficiency. In this paper, the CN stellate network consisted of
240 HH-like cells, simulated in NEURON and took approximately 90 seconds to run
a 80~ms stimulus on a 1.8 GHz CPU (32-bit Itanium, SGI Altix).  Evaluations the
AIV and IFR cost functions were a minor fraction of the total computational
time, being less than 3 seconds per network. The ST cost function was at a
considerable disadvantage because its evaluation took approximately 90 seconds,
which is similar to the simulation time. This could be improved because the
method for calculating the dynamic programming spike time distance was
sub-optimal.  On the 64-CPU SGI Altix, the amount of time required to run the GA
for 201 generations of 100 genomes took approximately 8 hours (a maximum of 40
CPUs were used at any point.  These computational loads are feasible in modern
systems and will enhance the development of more realistic BNN models.


\section{Conclusion}\label{sec:GA:conclusion}

The methods for generating experimentally relevant data are an important factor
when constraining a BNN model. In an ideal network model, where we can reproduce
the exact inputs to the network, as used in generating the training data, it
brings into question the validity of the training data to reproduce real
experiments.  Training data from an existing model, with target parameters
chosen randomly as performed in this paper, does not give us a representation of
a real network, but the development of methods for constraining new models is an
important step for generating microcircuits and larger networks.

\medskip{}

In this chapter, we have shown that the GA is an adequate method for parameter
optimization and that the ST and AIV cost functions are comparably good methods
for constraining BNNs. Further development is needed to enhance the robustness
of the cost function methods to input noise, especially for sensitivity and
robustness of inhibitory connections in the CN stellate network.
\todo[inline]{Last section you need to improve when you are done}


%%% Local Variables: 
%%% mode: latex
%%% TeX-master: "GAChapter"
%%% TeX-PDF-mode: nil
%%% End: 
