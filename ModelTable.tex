%% LaTeX2e file 
%% generated by the `filecontents' environment
%% from source `ltxtable' on 
%%
%\begin{table}[tp]
%  \centering
%\begin{center}
%\renewcommand{\tabularxcolumn}[1]{>\arraybackslash}b{#1}}
\begin{longtable}{p{1.2in}cX}%{p{0.25in}p{2.5in}}%{{\hsize=0.5\hsize}XcX}%{lcp{3.5in}}%
%{p{1.2in}c\p{3.5in}}
%{>{\hsize=\hsize/2}X>{\hsize=2\hsize}X}
%{p{1in}cp{2.7in}}%
\caption{CNSM model and simulation parameters}\label{tab:GA:GeneralParams} \\
%{\linewidth}{{X>{\hsize=.5\hsize}X>{\hsize=1.5\hsize}X}} %
\toprule  \multicolumn{1}{c}{\textbf{Parameter}}  &   \multicolumn{1}{c}{\textbf{Value}}  & \multicolumn{1}{c}{\textbf{Comment}} \\ \midrule 
\endfirsthead

\multicolumn{3}{c}%
{{\bfseries \tablename\ \thetable{} -- continued from previous page}} \\
\hline \multicolumn{1}{c}{\textbf{Parameter}} &
\multicolumn{1}{c}{\textbf{Value}} &
\multicolumn{1}{c}{\textbf{Comment}} \\ \midrule 
\endhead

\midrule \multicolumn{3}{c}{{Continued on next page}} \\ %\midrule
\endfoot
\bottomrule
\endlastfoot
%  \begin{tabularx}{\textwidth}{ccX}  

\multicolumn{2}{l}{\bf Auditory Model Parameters}  & Cat model, normal hearing    \citep{HeinzZhangEtAl:2001} \\ %\hline
%       Greenwood function for cats   (Hz)     & See Eq.~\ref{eq:GA:Greenwood}&\citep{Greenwood:1990} % $f=456.0\times 10^{\frac{x}{11.9} } -0.8$  & Basilar membrane position, $x$, and characteristic frequency, $f$, \citep{Greenwood:1990} \\ %\hline
            Channels             &                     60                     & Centre frequencies determined by Greenwood  function (See Eq.~\ref{eq:GA:Greenwood})\\
                Low Freq. (kHz)                &                   0.2                 & \\ %\hline
               High Freq. (kHz)                &                   30                  & \\ %\hline
%\begin{minipage}[l]{0.9in} % 
$s_{{\rm ANF}\to {\rm TS}}$,%\\ 
$s_{{\rm ANF}\to {\rm TV}}$ % 
%\end{minipage} 
\protect{(channels)} &  0     & All {\ANF} inputs to TS and TV cells come  from their own CF channel \\ %\hline
 $s_{{\rm ANF}\to {\rm DS}} $ \protect{(channels)}& \begin{minipage}[c]{0.9in}\begin{center} %
Above CF: 3 \\[-0.5ex]
Below CF: 6 %    
\end{center}\end{minipage}  & Approx. 1 octave above, 2 octaves below CF \citep{PalmerJiangEtAl:1996} \\ % \hline
   $s_{{\rm ANF}\to {\rm GLG}}$ \protect{(channels)} & 3  & \\ \midrule           
   \multicolumn{2}{l}{\bf ANF latency function}                &  \\ %\hline
      $A_{0}$ (ms) & 8.3& Eq.~\ref{eq:Methods:delay},  Cat model \citep{CarneyYin:1988} \\ %\hline 
 $A_1$ (cm)&6.49  &  \\ \midrule
\multicolumn{2}{l}{\bf Synaptic Delay adjustment} & {Conduction delay adjustment calculated from mean \FSL in \CN units } \\
\dANFTS ~(ms) &1.6& mean \FSL 3.6~ms \citep{RhodeSmith:1986}  \\
\dANFDS ~(ms) & 1.2& mean \FSL 3.2~ms \citep{RhodeSmith:1986}\\        
\dANFTV ~(ms) & 2.0& mean \FSL 4.0~ms \citep{OertelWickesberg:1993} \\
\dANFGLG ~(ms) & 2.3& 0.7~ms more than VCN core units \citep{FerragamoGoldingEtAl:1998}  \\ \midrule
\multicolumn{2}{l}{\bf Membrane Current Model Parameters}   &  \\ %\hline
             $C_m$ ($\mu$F/cm$^{2}$)         & 0.9 & Specific membrane capacitance   \\ %\hline
          Temperature     ($^\circ$C)      &       37       & \citet{RothmanManis:2003b} used 22$^\circ$C for their slice preparation. \\ %\hline
           Q$_{10}$             &            3            & Membrane current model temperature quality factor affects the activation and deactivation functions' time   constants. $Q=Q_{10}^{((37^\circ -22^\circ )/10)}$ \\ %\hline   
          $E_{\rm K}$    ~(mV)       &         -72         & Potassium reversal potential \\ %\hline
         $E_{\rm Na}$    ~(mV)       &          0          & Sodium reversal potential \\ %\hline
          $E_{\rm h}$    ~(mV)       &         -43         & Mixed-cation (Ih) reversal potential \\ %\hline
%           $\varphi$            &           0.5           & KLT variable \\ 
\midrule 
\multicolumn{2}{l}{\bf Synapse Parameters}       & \\ %\hline
     $\tau _{{\rm AMPA}}$  ~(ms)     &         0.36         & \VCN neurons, mature guinea pig \citep{GardnerTrussellEtAl:1999} \\ %\hline
         $E_{\rm Exc}$    ~(mV)      &          0           & Excitatory reversal potential\\[0.5ex] %\hline
 \begin{minipage}[l]{1in}%
$\tau _{{\rm Gly1}}$ (ms)\\ %[-0.5ex]
$\tau _{{\rm Gly2}}$ (ms)\end{minipage}  &    \begin{minipage}[c]{1in}\begin{center}%
0.4\\%[-0.5ex]
2.5   \end{center}
\end{minipage}         & {GlyR time-constants, MNTB neurons, mature guinea pig \citep{LeaoOleskevichEtAl:2004}} \\ %\hline
%     $\tau _{{\rm Gly2}}$  ~(ms)     &         2.5          & \\
 \begin{minipage}[l]{1in}%
$\tau _{{\rm GABA1}}$ (ms)\\[-0.5ex]
$\tau _{{\rm GABA2}}$ (ms)\end{minipage}     &    \begin{minipage}[c]{1in}\begin{center}%
0.7\\[-0.5ex]
9.0 \end{center}\end{minipage} & {\GABAa time-constants, MNTB neurons, mature guinea pig \citep{AwatramaniTurecekEtAl:2005}}\\
%     $\tau _{{\rm GABA1}}$ ~(ms)     &         0.7          & {\GABAa time-constants, MNTB neurons mature guinea pig \citep{AwatramaniTurecekEtAl:2005}}\\
%     $\tau _{{\rm GABA2}}$   ~(ms)   &          9           & \\ %\hline
        $E_{{\rm Inh}}$     ~(mV)     &         -75          & Chlorine reversal potential in Glycine and \GABAa receptors \\ %\hline
%           $\eta $           & \begin{minipage}[c]{1in}\begin{center}
% $\eta = \frac{1}{-\exp(t'/\tau_{Inh1})+\exp(t'/\tau_{inh2})}$ \\
% $t'=\frac{\tau_{Inh1}\tau_{Inh2}}{\tau_{Inh2}-\tau_{Inh1}} \ln(\tau_{Inh2}/\tau_{Inh1})$
% \end{center}   \end{minipage}     & Normalization factor for  double exponential synapse (\textit{exp2syn}) model \citep{HinesCarnevale:2000} \\ 
\midrule
\multicolumn{2}{l}{\bf NEURON Simulation Parameters} & \\ %\hline
        $dt$    ~(ms)     &          0.1           & Integration time step \\ %\hline
\textit{secondorder} &             2             & Crank-Nicholson method \\ %\hline
        $R$          &            25             & Stimulus repetition \\ %\hline
        $M$          &            240            & Total number of cells \\ %\hline
        $W$      ~(ms)    &          0.2         & PSTH bin width \\
%\bottomrule
%\end{tabularx}
%\end{table}
\end{longtable}
%\end{center}

%%% Local Variables: 
%%% mode: latex
%%% TeX-master: "GAChapter.tex"
%%% TeX-PDF-mode: nil
%%% End: 
